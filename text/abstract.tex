%% -*- TeX-engine: xetex; TeX-master: "draft.tex"; ispell-dictionary: "russian" -*-

\thispagestyle{empty}

\addtocontents{toc}{\protect\setcounter{tocdepth}{0}}
\section*{Реферат}

С.~\totalpages, рис.~\totalfigures, табл.~\totaltables.

В данной работе представлена вероятностная модель для анализа результатов
бисульфитного секвенирования. Предлагаемая модель является расширением
класса скрытых марковских моделей и называется переключающейся мультиномиальной
скрытой марковской моделью. У модели три состояния: одно из состояний соответствует
ошибкам бисульфитной конверсии, а другие два позволяют отличать области гиперметилирования
от областей гипометилирования. Модель предполагает, что вероятность перехода между
состояниями зависит от расстояния между моделируемыми наблюдениям. Использование
мультиномиального распределения для моделирования информации о покрытии
цитозинов позволяет учитывать наличие ошибок секвенирования. Реализация
модели была выполнена на языке Java в рамках биоинформатического проекта компании
JetBrains.

\textbf{Ключевые слова}: вероятностная модель, скрытая марковская модель,
эпигенетика, метилирование ДНК, бисульфитное секвенирование.

\addtocontents{toc}{\protect\setcounter{tocdepth}{2}}
