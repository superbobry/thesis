%% -*- TeX-engine: xetex; TeX-master: "draft.tex"; ispell-dictionary: "russian" -*-

\section{Обзор задачи}

\subsection{Биологические сведения}

\subsubsection{Метилирование ДНК}

Метилирование ДНК --- модификация цитозина в составе молекулы
ДНК, заключающаяся в присоединении метильной группы к C5-позиции цитозинового кольца
(рис.~\ref{fig:methylation}). Несмотря на то, что это не структурное изменение, при
делении клетки оно может сохраняться~\cite{Jones2012}.

\begin{figure}[h]
  \begin{center}
    \begingroup
    \setatomsep{1.8em}
    \chemname[1.8em]{\chemfig[line width=1pt]{*6([,,1]-\chembelow{N}{H}-(=O)-N=(-NH_2)-=)}}{цитозин}
    \setchemrel{1pt}{0pt}{8.5em}
    \chemrel{->,thick}
    \chemname[1.8em]{\chemfig[line width=1pt]{*6([,,1]-\chembelow{N}{H}-(=O)-N=(-NH_2)-(-CH_3)=)}}{5-метилцитозин}
    \endgroup
  \end{center}
  \caption{Метилирование ДНК}
  \label{fig:methylation}
\end{figure}

\subsubsection{Бисульфитное секвенирование}

Метилирование ДНК относится к классу эпигенетических модификаций, так как при присоединении метильной
группы к цитозину нуклеотидная структура ДНК не изменяется. Это обстоятельство мешает использовать
существующие методы анализа нуклеотидной последовательности для анализа метилирования. Решить
проблему можно с помощью химического механизма \emph{бисульфитной конверсии}, который позволяет
закодировать информацию о статусе метилирования в структуре ДНК. Суть механизма заключается в
следующем: под действием бисульфита натрия цитозин без метильной группы конвертируется в урацил;
5\hyp метилцитозин при этом не конвертируется.

Существуют различные методы анализа обработанной бисульфитом натрия ДНК \cite{Jones2012}, например,
полимеразная цепная реакция (ПЦР), ДНК\hyp микрочипы или секвенирование. Последний метод, называемый
\emph{бисульфитным секвенированием}, позволяет получить информацию о статусе метилирования каждого
цитозина в геноме, поэтому для полногеномного анализа метилирования используют именно его
\cite{lister2009human,Stadler2011,Xie2013,Ziller2013}.

В ходе секвенирования случайные фрагменты ДНК, обработанной бисульфитом натрия, читаются
секвенатором в объёме, достаточном для того, чтобы с большой вероятностью каждый
фрагмент был прочитан несколько раз. При этом урацил будет прочитан секвенатором как
тимин, а 5-метилцитозин --- как цитозин. Затем для каждого полученного прочтения ищется
соответствующий ему участок последовательности генома (рис.~\ref{fig:bsseq}). Обычно прочтения,
которым может соответствовать более одного участка в геноме, исключают из рассмотрения.

\begin{figure}[h]
  \centering
\begin{Verbatim}[commandchars=\\\{\}]
                 ↓            ↓      ↓
          CAAAAGA\bl{C}AAATAGTGATGT\bl{C}ACCAAT\bl{C}GAGC
          --------------------------------
               GA\bl{C}A         GT\bl{C}A  AAT\bl{T}
              AGA\bl{T}         TGT\bl{C}
               GA\bl{C}A   AGTG TGT\bl{C}    AT\bl{T}G
\end{Verbatim}
  \caption{Схематическое изображение выравнивания прочтений секвенатора (под чертой)
    на известную последовательность генома (над чертой). Стрелка (\texttt{↓}) указывает на
    метилированный цитозин.}
  \label{fig:bsseq}
\end{figure}

В качестве характеристики статуса метилирования конкретного цитозина в геноме обычно
используют \emph{уровень метилирования} ${\beta = \frac{\#C}{\#C + \#T} \in [0, 1]}$, где $\#C$ ---
количество прочтений, подтверждающих наличие метилирования, а $\#T$ --- количество прочтений,
опровергающих его. Можно считать, что $\beta$ --- оценка вероятности того, что случайно
взятое прочтение, покрывающее рассматриваемый цитозин, будет подтверждать метилирование.

Иногда вместо уровня метилирования также используют пару $(\#C, \#T)$ без каких-либо преобразований.

Протокол бисульфитного секвенирования, как уже было сказано, не исключает наличие в результатах
эксперимента ошибок. Кратко перечислим их основные источники.

\begin{itemize}
\item Недостаточная или избыточная обработка ДНК бисульфитом натрия приводит к
  ошибкам бисульфитной конверсии \cite{pmid18984622}, то есть переходам вида
  (цитозин $\rightarrow$ цитозин) и (5-метилцитозин $\rightarrow$ урацил).
\item Точечная мутация (цитозин $\rightarrow$ тимин) может ошибочно подтверждать наличие
  метилирования.
\item Ошибки секвенирования \cite{pmid22067484}, то есть неправильные прочтения секвенатором
  фрагментов ДНК, касающиеся цитозина и тимина, влияют на получаемые значения $\#C$ и $\#T$.
\end{itemize}

\subsection{Статистические сведения}

\subsubsection{Встречающиеся распределения}
\label{subsub:mentioned}

В тексте работы встречаются некоторые известные \cite{murphy2012machine} распределения. Ниже
приведены их функции плотности вероятности или функции вероятности в зависимости от области
определения.
\begin{itemize}
\item Биномиальное распределение:
  $$
  \op{Bin}(k; n, p) = \binom{n}{k} p^k (1 - p)^{n - k}
  \quad k \in \{0, \ldots, n\}; n \in \mathbb{N}_0, p \in [0, 1].
  $$
\item Мультиномиальное распределение:
  \begin{align*}
    \op{Mult}(\mathbf{k}; n, \mathbf{p})
    = \I{\sum_{i = 1}^d k_i = n} \frac{n!}{k_1! \dots k_d!} p_1^{k_1} \dots p_d^{k_d} \\
    k_i \in \{0, \ldots, n\}; n \in \mathbb{N}, p_i \in [0, 1], \sum\limits_{i = 1}^d p_i = 1,
  \end{align*}
  где $\I{x}$ --- индикаторная функция.
\item Дискретное (категориальное) распределение:
  $$
  \op{Cat}(k; \mathbf{p}) = p_1^{\I{k = 1}} \dots p_d^{\I{k = d}}
  \quad k \in \{1, \ldots, d\}; p_i \in [0, 1], \sum\limits_{i = 1}^d p_i = 1.
  $$
\item Распределение Гаусса (нормальное):
  $$
  \op{N}(x; \mu, \sigma^2)
  = \frac{1}{\sigma \sqrt{2 \pi}} \exp \left( - \frac{(x - \mu)^2}{2 \sigma^2} \right)
  \quad x \in \mathbb{R}; \mu \in \mathbb{R}, \sigma \in \mathbb{R}_{> 0}.
  $$
\item Бета-биномиальное распределение:
  \begin{align*}
  \op{BetaBin}(k; n, \alpha, \beta)
  = \binom{n}{k} \frac{\op{B}(k + \alpha, n - k + \beta)}{\op{B}(\alpha, \beta)} \\
  k \in \{0, \ldots, n\}; n \in \mathbb{N}_0, \alpha, \beta \in \mathbb{R}_{> 0}.
  \end{align*}
  Если $\alpha = \beta$, то распределение называется симметричным бета\hyp биномиальным.
\end{itemize}

\subsubsection{Математическое ожидание}

В работе используются следующие обозначения, связанные с математическим ожиданием.

\begin{itemize}
\item Математическое ожидание:
  $$
  \E{X}{f(X)} = \int f(x) p(x) dx,
  $$
  где $p(x)$ --- плотность распределения случайной величины $X$.
\item Условное математическое ожидание:
  $$
  \E{X|Y = y}{f(X)} = \int f(x) p(x|Y = y) dx,
  $$
  где $p(x|Y = y)$ -- плотность распределения случайной величины $X$ при
  условии того, что случайная величина $Y$ принимает значение $y$. В случаях, где это
  непротиворечиво, мы будем писать $\E{X|y}{f(X)}$ вместо $\E{X|Y = y}{f(X)}$.
\end{itemize}

\subsubsection{Информационный критерий Акаике}

Информационный критерий Акаике применяется для выбора из нескольких моделей, обученных на
одних и тех же данных методом максимального правдоподобия. Значение критерия для конкретной
модели вычисляется по формуле
$$
\op{AIC} = 2k - 2 \log L,
$$
где $k$ --- число параметров модели, а $L$ --- правдоподобие данных с точки зрения модели.
Критерий Акаике позволяет отранжировать модели по относительной предпочтительности: чем
ниже вычисленная величина, тем лучше модель. Критерий Акаике штрафует модели с большим
количеством параметров и тем самым защищает от переобучения \cite[С. 163-164]{murphy2012machine}.

\subsection{Существующие подходы}
\label{sub:existing}

\subsubsection{Скрытая марковская модель с Гауссовыми испусканиями}

Авторы \cite{Stadler2011} предлагают моделировать уровни метилирования с помощью скрытой
марковской модели \cite{Rabiner1989} с тремя состояниями: одно из состояний соответствует
ошибкам бисульфитной конверсии, а другие два позволяют отличать области гиперметилирования
от областей гипометилирования. Значение уровня метилирования моделируется с помощью распределения
Гаусса (см.~\ref{subsub:mentioned}). Обучение модели производится методом максимального
правдоподобия.

\paragraph{Анализ подхода}

Моделирование уровня метилирования, а не пары $(\#C, \#T)$, накладывает ограничения на
чувствительность модели, например, уровни метилирования $^5/_{10}$ и $^1/_2$ для модели
идентичны. Стоит также заметить, что распределение Гаусса определено на $\mathbb{R}$,
в то время как уровень метилирования --- величина из $[0, 1]$, то есть предлагаемая
авторами модель не соответствует моделируемым данным.

\subsubsection{MethylSeekR}

Одной из особенностей \cite{lister2009human,gaidatzis2014} метилирования в человеке является наличие
частично метилированных участков (partially methylated domains, PMDs)~---~длинных областей генома, в
которых распределение уровней метилирования стремится к равномерному. Авторы работы
\cite{Burger2013} предлагают трёхшаговую процедуру для нахождения регуляторных областей
гипометилирования в данных бисульфитного секвенирования человека с учётом наличия PMD.
Для этого геном сначала разбивается на непересекающиеся участки, содержащие 101
цитозин. Пары $(\#C, \#T)$ внутри одного участка моделируются с помощью симметричного
бета-биномиального распределения (см.~\ref{subsub:mentioned}), описываемого параметром
$\alpha$. Значение параметра $\alpha$ находится как апостериорное среднее. Вторым шагом
авторы предлагают моделировать последовательность $\alpha$ скрытой марковской моделью с
Гауссовыми испусканиями и двумя состояниями: участок является PMD или не является PMD.
Найденные таким образом частично метилированные участки исключаются из рассмотрения.
Гипометилированные области находятся среди оставшихся участков с помощью отсечки на уровень
метилирования, которая выбирается с помощью рандомизированной процедуры.

\paragraph{Анализ подхода}

Авторы не приводят мотивацию для выбора размера моделируемого участка, а также не
исследуют чувствительность метода к его размеру. Предположительно, с увеличением
размера участка распределение уровней метилирования будет стремиться к равномерному,
то есть $\alpha \approx 1$, что снизит чувствительность шага фильтрации PMD.

Аналогично предыдущему подходу, авторы используют распределение Гаусса для моделирования
положительной величины.

\subsubsection{MSC}
\label{subsub:msc}

В работе \cite{Cheng2014} предлагается алгоритм, позволяющий с некоторым ограниченным количеством
ошибок отделить цитозины без метильной группы от 5-метилцитозинов. Алгоритм основан на использовании
биномиальной смеси с двумя состояниями для моделирования пар $(\#C, \#T)$. Параметры смеси
оцениваются методом максимального правдоподобия.

\paragraph{Анализ подхода}

Основным результатом этой работы является использование подходов из теории кластерного
анализа и статистической проверки гипотез для контроля количества ошибок первого рода
(цитозин ошибочно назван метилированным) в результатах биномиальной смеси.

\subsection{Экспериментальные данные}
\label{sub:data}

NCBI GEO\footnote{\url{http://www.ncbi.nlm.nih.gov/geo}} --- крупнейший открытый архив экспериментальных
данных, полученных из биологических экспериментов. Эксперименты внутри NCBI GEO объединены в серии. У
каждой серии экспериментов есть свой уникальный номер, по которому можно найти описание протокола,
использованного для получения данных, последовательность этапов пост-обработки данных,
названия связанных статей и так далее.

Для оценки моделей в данной работе используются данные бисульфитного секвенирования стволовых клеток
мыши (\emph{M. musculus}), описанные в статье \cite{Stadler2011}, номер серии экспериментов GSE30202.
