%% -*- TeX-engine: xetex; TeX-master: "draft.tex"; ispell-dictionary: "russian" -*-

\section*{Заключение}

В ходе проделанной работы достигнуты следующие результаты.

\begin{enumerate}
\item Проанализированы и описаны существующие подходы к обработке и моделированию данных бисульфитного
  секвенирования: скрытая марковская модель с Гауссовыми испусканиями, MethylSeekR и MSC.
  (См.~\ref{sub:existing})
\item Разработаны несколько вероятностных моделей: биномиальная смесь, биномиальная скрытая
  марковская модель, переключающаяся биномиальная скрытая марковская модель, переключающаяся
  мультиномиальная скрытая марковская модель. Для каждой модели приведены мотивация, описание и
  обоснование. (См.~\ref{sec:models})
\item В качестве наиболее подходящей модели выбрана переключающаяся мультиномиальная скрытая марковская
  модель, так как она позволяет учитывать особенности моделируемых данных и показала хорошее время
  обучения: 22 минуты на самой большой хромосоме генома мыши. (См.~\ref{sub:implementation})
\item Проведено сравнение переключающейся мультиномиальной скрытой марковской модели с алгоритмом MSC на
  данных бисульфитного секвенирования стволовых клеток мыши. Разработанная модель учла 100\%
  результатов, показанных MSC, а также нашла другие метилированные цитозины.
  (См.~\ref{sub:comparison})
\item Продемонстирована работа модели на генах <<домашнего хозяйства>> мыши. На восьми из десяти
  генов модель показала ожидаемые результаты. (См.~\ref{sub:housekeeping})
\end{enumerate}

В качестве развития модели предполагается учёт точечных мутаций в исследуемых образцах. Кроме того,
планируется обобщение на сравнение нескольких биологических образцов для исследования изменений
метилирования, например, при раке или дифференцировке клетки. Описанную модель также можно расширить
для учёта асимметрии в метилировании соседних цитозинов на разных цепочках ДНК.

Полученная модель уже используется в биоинформатическом проекте компании JetBrains для разметки геномов в
соответствии с уровнем метилирования. Также результаты, описанные в работе, будут оформлены в
статью.
