%% -*- TeX-engine: xetex; TeX-master: "draft.tex"; ispell-dictionary: "russian" -*-

\section*{Введение}

ДНК (дезоксирибонуклеиновая кислота) --- длинная двухцепочечная молекула, являющаяся носителем
генетической информации в биологических организмах. Совокупность ДНК биологического организма
называют геномом. Основу цепочки ДНК составляют четыре нуклеотидных остатка: аденин, гуанин, цитозин и
тимин.

% TODO: эпигенетика
% С нуклеотидами в составе молекулы ДНК могут происходить генетические и эпигенетические
% изменения. Первая группа включает в себя изменения которые изменяют последовательность нуклеотидов
% в цепочке, например, замена одного нуклеотида на другой или удаление нуклеотида. Изменения во второй
% группе не изменяют структуру ДНК.

Присоединение метильной группы ($\op{CH}_3$) к цитозину в составе молекулы ДНК является
одним из ключевых механизмов регуляции жизни клетки \cite{Smith2013}. Это и есть
метилирование ДНК. Одним из методов экспериментальной оценки метилирования является
бисульфитное секвенирование. Суть метода заключается в обработке ДНК бисульфитом натрия с
последующим секвенированием, то есть чтением нуклеотидной последовательности.

Мотивацией для анализа метилирования послужили исследования
\cite{lister2009human,Stadler2011,Xie2013}, авторы которых изучали данные бисульфитного
секвенирования человека и мыши и пришли к выводу, что области генома, редко содержащие
цитозины с метильной группой, представляют особый интерес для изучения. В частности,
в работе \cite{Stadler2011} было показано, что области гипометилирования\footnote{
Цитозин называется \emph{гипометилированым}, если к нему редко присоединяется метильная группа.
Обратная ситуация называется \emph{гиперметилированием}, в этом случае метильная группа почти
всегда присутствует.} находятся в похожем состоянии у различных видов, а также оказывают
влияние на находящиеся рядом функциональные участки генома. Изменение метилирования
влияет на эти участки, тем самым вызывая нарушения в жизнедеятельности клетки, например,
приводя к раку \cite{Xie2013}.

Для получения информации о строении и внутренних процессах клетки ставятся
биологические эксперименты, например, бисульфитное секвенирование. Во время
этих экспериментов мы не можем обеспечить изоляцию от окружающей среды, а
также идеальную точность измерительных приборов. Из-за этого в результатах
экспериментов появляется шум, и последующий анализ получается неточным. Частично
эту проблему решает использование вероятностных моделей. Возвращаясь к бисульфитному
секвенированию, можно отметить, что моделей для него почти нет, а существующие,
скорее всего, можно улучшить.

Цель данной работы --- разработать математическую модель, позволяющую качественно
описывать результаты бисульфитного секвенирования.

Для достижения цели были поставлены следующие задачи.

\begin{enumerate}
\item Проанализировать существующие подходы к обработке и моделированию данных бисульфитного
  секвенирования.
\item Предложить, обосновать и реализовать несколько вероятностных моделей.
\item Сформулировать критерии отбора и выбрать наиболее подходящую модель.
\item Сравнить выбранную модель с уже существующими.
\end{enumerate}
